%%%%%%%%%%%%%%%%%%%%%%%%%%%%%%%%%%%%%%%%%%%%%%%%%%%%%%%%%%%%%%%%%%%%%%%%%%%%%%%%
%                               Pandoc preamble                                %
%%%%%%%%%%%%%%%%%%%%%%%%%%%%%%%%%%%%%%%%%%%%%%%%%%%%%%%%%%%%%%%%%%%%%%%%%%%%%%%%
% @author: shawshary
% @date: 2022-10-23
% @brief: used by pandoc -H option when generating pdf files.
%%%%%%%%%%%%%%%%%%%%%%%%%%%%%%%%%%%%%%%%%%%%%%%%%%%%%%%%%%%%%%%%%%%%%%%%%%%%%%%%

\usepackage[utf8]{inputenc}

% English font support.
\usepackage{fontspec}
\setmainfont{Times New Roman}

% Chinese font support.
\usepackage{xeCJK}
\setCJKmainfont[BoldFont=SimHei,ItalicFont=KaiTi]{SimSun}

% Geometry.
\usepackage[top=2cm, bottom=2cm, left=2cm, right=2cm]{geometry}

% Graphic.
\usepackage{graphicx}
%\graphicspath{{/home/xinyu/Media/images/}}

% Fix figure position.
\usepackage{float}
\floatplacement{figure}{H}

% color box
\usepackage[most]{tcolorbox}
% Change the style of quote
\tcolorboxenvironment{quote}{
  enhanced,
  breakable,
  boxrule=1pt,
  sharp corners=all,
}


% indent the first paragraph right after the section.
\usepackage{indentfirst}

% Paragraph indent and skip.
\setlength{\parindent}{2em}
\setlength{\parskip}{1.5em}

% 设置行距为默认行距的1.3倍。
\renewcommand{\baselinestretch}{1.3}

% Set the title of tables and figures to chinese format.
\renewcommand{\figurename}{图}
\renewcommand{\tablename}{表}

% Header and footnotes.
\usepackage{fancyhdr}
\pagestyle{fancy}
%\fancyhf{}
\fancyhead[LE,LO]{{\textit{Copyright 2022}}}
%\fancyhead[RE,RO]{\includegraphics[width=1.5cm]{../image.png}}
\fancyfoot[CE,CO]{\thepage}

